%%*************************************************************************
%% Legal Notice:
%% This code is offered as-is without any warranty either expressed or
%% implied; without even the implied warranty of MERCHANTABILITY or
%% FITNESS FOR A PARTICULAR PURPOSE! 
%% User assumes all risk.
%% In no event shall IEEE or any contributor to this code be liable for
%% any damages or losses, including, but not limited to, incidental,
%% consequential, or any other damages, resulting from the use or misuse
%% of any information contained here.
%%
%% All comments are the opinions of their respective authors and are not
%% necessarily endorsed by the IEEE.
%%
%% This work is distributed under the LaTeX Project Public License (LPPL)
%% ( http://www.latex-project.org/ ) version 1.3, and may be freely used,
%% distributed and modified. A copy of the LPPL, version 1.3, is included
%% in the base LaTeX documentation of all distributions of LaTeX released
%% 2003/12/01 or later.
%% Retain all contribution notices and credits.
%% ** Modified files should be clearly indicated as such, including  **
%% ** renaming them and changing author support contact information. **
%%
%% File list of work: IEEEtran.cls, IEEEtran_HOWTO.pdf, bare_adv.tex,
%%                    bare_conf.tex, bare_jrnl.tex, bare_jrnl_compsoc.tex,
%%                    bare_jrnl_transmag.tex
%%*************************************************************************

% Note that the a4paper option is mainly intended so that authors in
% countries using A4 can easily print to A4 and see how their papers will
% look in print - the typesetting of the document will not typically be
% affected with changes in paper size (but the bottom and side margins will).
% Use the testflow package mentioned above to verify correct handling of
% both paper sizes by the user's LaTeX system.
%
% Also note that the "draftcls" or "draftclsnofoot", not "draft", option
% should be used if it is desired that the figures are to be displayed in
% draft mode.
%
\documentclass[journal]{IEEEtran}
%
% If IEEEtran.cls has not been installed into the LaTeX system files,
% manually specify the path to it like:
% \documentclass[journal]{../sty/IEEEtran}





% Some very useful LaTeX packages include:
% (uncomment the ones you want to load)


% *** MISC UTILITY PACKAGES ***
%
%\usepackage{ifpdf}
% Heiko Oberdiek's ifpdf.sty is very useful if you need conditional
% compilation based on whether the output is pdf or dvi.
% usage:
% \ifpdf
%   % pdf code
% \else
%   % dvi code
% \fi
% The latest version of ifpdf.sty can be obtained from:
% http://www.ctan.org/tex-archive/macros/latex/contrib/oberdiek/
% Also, note that IEEEtran.cls V1.7 and later provides a builtin
% \ifCLASSINFOpdf conditional that works the same way.
% When switching from latex to pdflatex and vice-versa, the compiler may
% have to be run twice to clear warning/error messages.






% *** CITATION PACKAGES ***
%
%\usepackage{cite}
% cite.sty was written by Donald Arseneau
% V1.6 and later of IEEEtran pre-defines the format of the cite.sty package
% \cite{} output to follow that of IEEE. Loading the cite package will
% result in citation numbers being automatically sorted and properly
% "compressed/ranged". e.g., [1], [9], [2], [7], [5], [6] without using
% cite.sty will become [1], [2], [5]--[7], [9] using cite.sty. cite.sty's
% \cite will automatically add leading space, if needed. Use cite.sty's
% noadjust option (cite.sty V3.8 and later) if you want to turn this off
% such as if a citation ever needs to be enclosed in parenthesis.
% cite.sty is already installed on most LaTeX systems. Be sure and use
% version 4.0 (2003-05-27) and later if using hyperref.sty. cite.sty does
% not currently provide for hyperlinked citations.
% The latest version can be obtained at:
% http://www.ctan.org/tex-archive/macros/latex/contrib/cite/
% The documentation is contained in the cite.sty file itself.






% *** GRAPHICS RELATED PACKAGES ***
%
\ifCLASSINFOpdf
  % \usepackage[pdftex]{graphicx}
  % declare the path(s) where your graphic files are
  % \graphicspath{{../pdf/}{../jpeg/}}
  % and their extensions so you won't have to specify these with
  % every instance of \includegraphics
  % \DeclareGraphicsExtensions{.pdf,.jpeg,.png}
\else
  % or other class option (dvipsone, dvipdf, if not using dvips). graphicx
  % will default to the driver specified in the system graphics.cfg if no
  % driver is specified.
  % \usepackage[dvips]{graphicx}
  % declare the path(s) where your graphic files are
  % \graphicspath{{../eps/}}
  % and their extensions so you won't have to specify these with
  % every instance of \includegraphics
  % \DeclareGraphicsExtensions{.eps}
\fi
% graphicx was written by David Carlisle and Sebastian Rahtz. It is
% required if you want graphics, photos, etc. graphicx.sty is already
% installed on most LaTeX systems. The latest version and documentation
% can be obtained at: 
% http://www.ctan.org/tex-archive/macros/latex/required/graphics/
% Another good source of documentation is "Using Imported Graphics in
% LaTeX2e" by Keith Reckdahl which can be found at:
% http://www.ctan.org/tex-archive/info/epslatex/
%
% latex, and pdflatex in dvi mode, support graphics in encapsulated
% postscript (.eps) format. pdflatex in pdf mode supports graphics
% in .pdf, .jpeg, .png and .mps (metapost) formats. Users should ensure
% that all non-photo figures use a vector format (.eps, .pdf, .mps) and
% not a bitmapped formats (.jpeg, .png). IEEE frowns on bitmapped formats
% which can result in "jaggedy"/blurry rendering of lines and letters as
% well as large increases in file sizes.
%
% You can find documentation about the pdfTeX application at:
% http://www.tug.org/applications/pdftex





% *** MATH PACKAGES ***
%
%\usepackage[cmex10]{amsmath}
% A popular package from the American Mathematical Society that provides
% many useful and powerful commands for dealing with mathematics. If using
% it, be sure to load this package with the cmex10 option to ensure that
% only type 1 fonts will utilized at all point sizes. Without this option,
% it is possible that some math symbols, particularly those within
% footnotes, will be rendered in bitmap form which will result in a
% document that can not be IEEE Xplore compliant!
%
% Also, note that the amsmath package sets \interdisplaylinepenalty to 10000
% thus preventing page breaks from occurring within multiline equations. Use:
%\interdisplaylinepenalty=2500
% after loading amsmath to restore such page breaks as IEEEtran.cls normally
% does. amsmath.sty is already installed on most LaTeX systems. The latest
% version and documentation can be obtained at:
% http://www.ctan.org/tex-archive/macros/latex/required/amslatex/math/





% *** SPECIALIZED LIST PACKAGES ***
%
%\usepackage{algorithmic}
% algorithmic.sty was written by Peter Williams and Rogerio Brito.
% This package provides an algorithmic environment fo describing algorithms.
% You can use the algorithmic environment in-text or within a figure
% environment to provide for a floating algorithm. Do NOT use the algorithm
% floating environment provided by algorithm.sty (by the same authors) or
% algorithm2e.sty (by Christophe Fiorio) as IEEE does not use dedicated
% algorithm float types and packages that provide these will not provide
% correct IEEE style captions. The latest version and documentation of
% algorithmic.sty can be obtained at:
% http://www.ctan.org/tex-archive/macros/latex/contrib/algorithms/
% There is also a support site at:
% http://algorithms.berlios.de/index.html
% Also of interest may be the (relatively newer and more customizable)
% algorithmicx.sty package by Szasz Janos:
% http://www.ctan.org/tex-archive/macros/latex/contrib/algorithmicx/




% *** ALIGNMENT PACKAGES ***
%
%\usepackage{array}
% Frank Mittelbach's and David Carlisle's array.sty patches and improves
% the standard LaTeX2e array and tabular environments to provide better
% appearance and additional user controls. As the default LaTeX2e table
% generation code is lacking to the point of almost being broken with
% respect to the quality of the end results, all users are strongly
% advised to use an enhanced (at the very least that provided by array.sty)
% set of table tools. array.sty is already installed on most systems. The
% latest version and documentation can be obtained at:
% http://www.ctan.org/tex-archive/macros/latex/required/tools/


% IEEEtran contains the IEEEeqnarray family of commands that can be used to
% generate multiline equations as well as matrices, tables, etc., of high
% quality.




% *** SUBFIGURE PACKAGES ***
%\ifCLASSOPTIONcompsoc
%  \usepackage[caption=false,font=normalsize,labelfont=sf,textfont=sf]{subfig}
%\else
%  \usepackage[caption=false,font=footnotesize]{subfig}
%\fi
% subfig.sty, written by Steven Douglas Cochran, is the modern replacement
% for subfigure.sty, the latter of which is no longer maintained and is
% incompatible with some LaTeX packages including fixltx2e. However,
% subfig.sty requires and automatically loads Axel Sommerfeldt's caption.sty
% which will override IEEEtran.cls' handling of captions and this will result
% in non-IEEE style figure/table captions. To prevent this problem, be sure
% and invoke subfig.sty's "caption=false" package option (available since
% subfig.sty version 1.3, 2005/06/28) as this is will preserve IEEEtran.cls
% handling of captions.
% Note that the Computer Society format requires a larger sans serif font
% than the serif footnote size font used in traditional IEEE formatting
% and thus the need to invoke different subfig.sty package options depending
% on whether compsoc mode has been enabled.
%
% The latest version and documentation of subfig.sty can be obtained at:
% http://www.ctan.org/tex-archive/macros/latex/contrib/subfig/




% *** FLOAT PACKAGES ***
%
%\usepackage{fixltx2e}
% fixltx2e, the successor to the earlier fix2col.sty, was written by
% Frank Mittelbach and David Carlisle. This package corrects a few problems
% in the LaTeX2e kernel, the most notable of which is that in current
% LaTeX2e releases, the ordering of single and double column floats is not
% guaranteed to be preserved. Thus, an unpatched LaTeX2e can allow a
% single column figure to be placed prior to an earlier double column
% figure. The latest version and documentation can be found at:
% http://www.ctan.org/tex-archive/macros/latex/base/


%\usepackage{stfloats}
% stfloats.sty was written by Sigitas Tolusis. This package gives LaTeX2e
% the ability to do double column floats at the bottom of the page as well
% as the top. (e.g., "\begin{figure*}[!b]" is not normally possible in
% LaTeX2e). It also provides a command:
%\fnbelowfloat
% to enable the placement of footnotes below bottom floats (the standard
% LaTeX2e kernel puts them above bottom floats). This is an invasive package
% which rewrites many portions of the LaTeX2e float routines. It may not work
% with other packages that modify the LaTeX2e float routines. The latest
% version and documentation can be obtained at:
% http://www.ctan.org/tex-archive/macros/latex/contrib/sttools/
% Do not use the stfloats baselinefloat ability as IEEE does not allow
% \baselineskip to stretch. Authors submitting work to the IEEE should note
% that IEEE rarely uses double column equations and that authors should try
% to avoid such use. Do not be tempted to use the cuted.sty or midfloat.sty
% packages (also by Sigitas Tolusis) as IEEE does not format its papers in
% such ways.
% Do not attempt to use stfloats with fixltx2e as they are incompatible.
% Instead, use Morten Hogholm'a dblfloatfix which combines the features
% of both fixltx2e and stfloats:
%
% \usepackage{dblfloatfix}
% The latest version can be found at:
% http://www.ctan.org/tex-archive/macros/latex/contrib/dblfloatfix/




%\ifCLASSOPTIONcaptionsoff
%  \usepackage[nomarkers]{endfloat}
% \let\MYoriglatexcaption\caption
% \renewcommand{\caption}[2][\relax]{\MYoriglatexcaption[#2]{#2}}
%\fi
% endfloat.sty was written by James Darrell McCauley, Jeff Goldberg and 
% Axel Sommerfeldt. This package may be useful when used in conjunction with 
% IEEEtran.cls'  captionsoff option. Some IEEE journals/societies require that
% submissions have lists of figures/tables at the end of the paper and that
% figures/tables without any captions are placed on a page by themselves at
% the end of the document. If needed, the draftcls IEEEtran class option or
% \CLASSINPUTbaselinestretch interface can be used to increase the line
% spacing as well. Be sure and use the nomarkers option of endfloat to
% prevent endfloat from "marking" where the figures would have been placed
% in the text. The two hack lines of code above are a slight modification of
% that suggested by in the endfloat docs (section 8.4.1) to ensure that
% the full captions always appear in the list of figures/tables - even if
% the user used the short optional argument of \caption[]{}.
% IEEE papers do not typically make use of \caption[]'s optional argument,
% so this should not be an issue. A similar trick can be used to disable
% captions of packages such as subfig.sty that lack options to turn off
% the subcaptions:
% For subfig.sty:
% \let\MYorigsubfloat\subfloat
% \renewcommand{\subfloat}[2][\relax]{\MYorigsubfloat[]{#2}}
% However, the above trick will not work if both optional arguments of
% the \subfloat command are used. Furthermore, there needs to be a
% description of each subfigure *somewhere* and endfloat does not add
% subfigure captions to its list of figures. Thus, the best approach is to
% avoid the use of subfigure captions (many IEEE journals avoid them anyway)
% and instead reference/explain all the subfigures within the main caption.
% The latest version of endfloat.sty and its documentation can obtained at:
% http://www.ctan.org/tex-archive/macros/latex/contrib/endfloat/
%
% The IEEEtran \ifCLASSOPTIONcaptionsoff conditional can also be used
% later in the document, say, to conditionally put the References on a 
% page by themselves.




% *** PDF, URL AND HYPERLINK PACKAGES ***
%
%\usepackage{url}
% url.sty was written by Donald Arseneau. It provides better support for
% handling and breaking URLs. url.sty is already installed on most LaTeX
% systems. The latest version and documentation can be obtained at:
% http://www.ctan.org/tex-archive/macros/latex/contrib/url/
% Basically, \url{my_url_here}.




% *** Do not adjust lengths that control margins, column widths, etc. ***
% *** Do not use packages that alter fonts (such as pslatex).         ***
% There should be no need to do such things with IEEEtran.cls V1.6 and later.
% (Unless specifically asked to do so by the journal or conference you plan
% to submit to, of course. )


% correct bad hyphenation here
\hyphenation{optical net-works semiconductor}

\usepackage[utf8]{inputenc}
\usepackage{url}
\usepackage{breqn}

\begin{document}

%
% paper title
% can use linebreaks \\ within to get better formatting as desired
% Do not put math or special symbols in the title.
\title{Geolocalización de Nodos Inalámbricos}
%
%
% author names and IEEE memberships
% note positions of commas and nonbreaking spaces ( ~ ) LaTeX will not break
% a structure at a ~ so this keeps an author's name from being broken across
% two lines.
% use \thanks{} to gain access to the first footnote area
% a separate \thanks must be used for each paragraph as LaTeX2e's \thanks
% was not built to handle multiple paragraphs
%

\author{Matías Parodi, Guillermo Reisch}

% note the % following the last \IEEEmembership and also \thanks - 
% these prevent an unwanted space from occurring between the last author name
% and the end of the author line. i.e., if you had this:
% 
% \author{....lastname \thanks{...} \thanks{...} }
%                     ^------------^------------^----Do not want these spaces!
%
% a space would be appended to the last name and could cause every name on that
% line to be shifted left slightly. This is one of those "LaTeX things". For
% instance, "\textbf{A} \textbf{B}" will typeset as "A B" not "AB". To get
% "AB" then you have to do: "\textbf{A}\textbf{B}"
% \thanks is no different in this regard, so shield the last } of each \thanks
% that ends a line with a % and do not let a space in before the next \thanks.
% Spaces after \IEEEmembership other than the last one are OK (and needed) as
% you are supposed to have spaces between the names. For what it is worth,
% this is a minor point as most people would not even notice if the said evil
% space somehow managed to creep in.



% The paper headers
\markboth{Journal of \LaTeX\ Class Files,~Vol.~11, No.~4, December~2012}%
{Shell \MakeLowercase{\textit{et al.}}: Bare Demo of IEEEtran.cls for Journals}
% The only time the second header will appear is for the odd numbered pages
% after the title page when using the twoside option.
% 
% *** Note that you probably will NOT want to include the author's ***
% *** name in the headers of peer review papers.                   ***
% You can use \ifCLASSOPTIONpeerreview for conditional compilation here if
% you desire.




% If you want to put a publisher's ID mark on the page you can do it like
% this:
%\IEEEpubid{0000--0000/00\$00.00~\copyright~2012 IEEE}
% Remember, if you use this you must call \IEEEpubidadjcol in the second
% column for its text to clear the IEEEpubid mark.



% use for special paper notices
%\IEEEspecialpapernotice{(Invited Paper)}




% make the title area
\maketitle

% As a general rule, do not put math, special symbols or citations
% in the abstract or keywords.

% FIXME
%\begin{abstract}
%The abstract goes here.
%\end{abstract}

% Note that keywords are not normally used for peerreview papers.
\begin{IEEEkeywords}
Algoritmos Evolutivos, triangulación, WiFi
\end{IEEEkeywords}






% For peer review papers, you can put extra information on the cover
% page as needed:
% \ifCLASSOPTIONpeerreview
% \begin{center} \bfseries EDICS Category: 3-BBND \end{center}
% \fi
%
% For peerreview papers, this IEEEtran command inserts a page break and
% creates the second title. It will be ignored for other modes.
\IEEEpeerreviewmaketitle


\section{Introducción}
% The very first letter is a 2 line initial drop letter followed
% by the rest of the first word in caps.
% 
% form to use if the first word consists of a single letter:
% \IEEEPARstart{A}{demo} file is ....
% 
% form to use if you need the single drop letter followed by
% normal text (unknown if ever used by IEEE):
% \IEEEPARstart{A}{}demo file is ....
% 
% Some journals put the first two words in caps:
% \IEEEPARstart{T}{his demo} file is ....
% 
% Here we have the typical use of a "T" for an initial drop letter
% and "HIS" in caps to complete the first word.
\IEEEPARstart{E}{l} presente informe consiste en la evaluación del comportamiento de un algoritmo evolutivo aplicado a la triangulación de dispositivos móviles en relación a la posición de al menos tres routers inalámbricos, cuyas posiciones absolutas o relativas a un centro arbitrario sí se conocen.\\

Se asumirá que los routers presentes en el entorno comparten características similares (e.g. alcance máximo), lo cual resulta razonable asumir para un gran número de routers WiFi existentes en el mercado. Esto será considerado al momento de la inicialización de la población inicial.\\

Por otro lado, resulta inapropiado asumir que las señales no se ven afectadas por obstáculos encontrados en el entorno, lo cual obligará a estimar, además de la posición, cuál es la caída o atenuación de la señal hacia cada router y para esto se estudiarán varias alternativas de modelo de propagación de ondas.\\

\hfill Julio 19, 2013

\section{Problema}

Debido a la explosión de las redes WiFi en todo el mundo \cite{wifi:coverage}, en los últimos años han surgido servicios de geolocalización por triangulación WiFi \cite{wifi:positioning} que ofrecen APIs \cite{google:maps} y SKDs  \cite{skyhook:location} para el desarrollo de aplicaciones sensibles a la posición del usuario.\\

Desafortunadamente, ninguna de las grandes empresas en estas áreas han publicado especificaciones de sus implementaciones, por lo que se desconocen muchos detalles del funcionamiento interno y en particular de los algoritmos que utilizan.\\

En el ámbito científico se exploraron varias alternativas. Quizás el enfoque más sencillo consiste en la triangulación de nodos a partir simplemente de las intensidades de las señales a cada router inalámbrico, como se analiza en \cite{simple:triangulation}. Este enfoque presenta varias desventajas ya que no considera muchos aspectos de la realidad como la caída de señal por ruido electromagnético u obstáculos en el ambiente, condiciones climatológicas que afectan la señal, diversidad de antenas, etc.\\

Otro grupo de investigadores han explorado el problema desde el área de machine learning, consiguiendo buenos resultados utilizando dos enfoques distintos. Por un lado, se han implementado satisfactoriamente sistemas basados en \textit{fuzzy logic} \cite{fuzzy:logic:based:system}, mientras que otros han intentado la integración de más de un sistema de triangulación, con el fin de tomar los beneficios de cada uno de ellos, combinando los resultados a partir de constantes de peso aprendidas y que se van ajustando a medida que se ejecuta el algoritmo \cite{adaptive:weighting}.\\

Dado el tamaño del espacio de búsqueda, resulta interesante atacar el problema mediante un algoritmo evolutivo con el objetivo de lograr aproximar la posición con una precisión comparable a los servicios existentes pero para eso habrá que restringir el problema de lo contrario sería imposible abordarlo dada la cantidad de variables libres presentes.\\

\section{Dificultades}

El nivel de señal recibido por el dispositivo que se intenta triangular varía con el tiempo, aunque no hayan cambiados sus posiciones relativas. Es importante entender qué ocasiona dichas fluctuaciones, ya que tener una buena estimación de la señal será esencial para la correcta evaluación del individuo.\\

Por mayores detalles respecto a cuales son los factores más importantes que afectan a la señal recibida y como estos influyen en la misma refererse a \cite{fuzzy:logic:based:system} y \cite{radar:tracking:system}.\\

\subsection{Tecnología}

Los cuatro factores más importantes respecto a los dispositivos con los que se trabaja son: la frecuencia de transmisión, el tipo de antena, la ganancia de la antena y la potencia de transmisión.\\

A pesar de que existen muchas marcas y modelos existentes en el mercado, los fabricantes de routers modernos respetan estándares internacionales para garantizar el correcto funcionamiento de redes WiFi con cualquier dispositivo inalámbrico.\\

Mediante un estudio de mercado se encontró que los routers domésticos son bastante homogéneos en cuanto a sus características, lo cual nos permite asumir algunas cosas que ayudarán a simplificar el el problema de manera de hacerlo abordable.\\

La frecuencia de transmisión depende del canal en el que se encuentre transmitiendo el router, sin embargo el rango en el que pueden estar transmitiendo no es demasiado grande, pudiendo variar entre 2,412GHz a 2,472GHz. Para simplificar el problema se asumirá entonces que el router opera en una frecuencia similar al promedio, 2,442GHz, correspondiente al canal 7.\\

En cuanto al tipo de antena, se encontró que todos los routers modernos tienen entre una y dos antenas isotrópicas de ganancia 5dBm, que emiten uniformemente hacia todos lados, por lo que la energía se disipará en el ambiente proporcionalmente al cuadrado de la distancia.\\

En cuanto al a potencia de transmisión, muchos routers permiten configurarla, pudiendo variar de 1dBm (1mW) a 20dBm (100mW), sin embargo en la práctica la mayoría de routers usan la potencia máxima con el fin de hacer llegar la señal más lejos.\\

\subsection{Entorno}

El entorno puede afectar significativamente la calidad e intensidad de la señal, en particular el ruido electromagnético, condiciones climáticas y obstáculos presentes ocasionan una pérdida significativa de la misma.\\

Dado que es imposible construir un modelo de propagación para cada entorno, ya que este afecta de manera diferente a cada dispositivo, resulta vital considerar la caída de la señal percibida por el dispositivo móvil, en lugar de intentar modelar cada variable por separado.\\

Como se verá más adelante, cada router tendrá asociada una caída de señal, la cual el algoritmo intentará aproximar de manera de encontrar la posición más probable del individuo en relación a ellos.\\

Se tendrá entonces que la señal real será igual a la señal emitida por el router, menos la pérdida por disipación (dependiente de la distancia), menos la pérdida por obstáculos (dependiente del entorno).\\

\subsection{Muestreo}

Una tarjeta WiFi estándar solo puede escuchar un canal por vez, por lo que para obtener todas las redes visibles en un entorno es necesario iterar en todos los canales en busca del anuncio de los routers inalámbricos.\\

Esto ocasiona que la información obtenida sobre una red en un determinado momento esté desactualizada por algunos segundos. Esto, sumado a los factores antes mencionados, ocasiona que mediones consecutivas pueden dar valores de intensidad de señal relativamente distintos.\\

Para evitar este problema se tomará como intensidad de la señal hacia un router el promedio de las cinco últimas lecturas.\\

\subsection{Cantidad de Nodos}

La cantidad de nodos afectará la precisión con la que se consiga triangular al dispositivo móvil pero serán necesarios al menos 3 routers para un correcto posicionamiento del mismo.\\

\section{Modelos existentes}

El modelo de propagación de ondas permitirá la traducción de una intensidad de señal a una distancia y viceversa. Se usará fuertemente este modelo para evaluar al individuo y encontrar el error entre la estimación de la posición y la distancia teórica real.\\

Existen muchas aproximaciones que fueron analizadas en \cite{low:cost:location:determination} y \cite{generic:model:signal:propagation}, aquí se describirán 4 de ellas, partiendo de la más simple a la más compleja.\\

\subsection{Empirical Model}

Este modelo de propagación es quizás el más preciso de todos a pesar de ser muy simple. La principal desventaja, que lo hace impracticable en muchos casos, es que requiere del estudio del ambiente en cuestión pero como vimos antes esto depende fuertemente de las condiciones del ambiente en un instante dado, por lo que su utilidad es limitada a medida que avanza el tiempo.\\

$$distance(rs) = |rs - ts| \times k$$

En donde $rs$ es la intensidad de señal recibida, $ts$ es la intensidad de señal transmitida por el router y $k$ es un valor empírico que corresponde a la distancia cubierta por 1dBm.\\

\subsection{Land Propagation}
\subsection{Generic Radio Signal Propagation Model}
\subsection{Empirical Signal Propagation Model (WAF)}

\subsection{Free Space Propagation}

Este modelo de propagación está inspirado en la ecuación de propagación de Friis \cite{friis:transmission:equation} y es quizás el más útil de todos.

$$P_r = \frac {G_t \times G_r \times \lambda^2} {(4 \times \pi \times d)^2} \times P_t$$

En donde $P_r$ es la potencia recibida, $G_t$ y $G_r$ son la ganancia de las antenas de transmisión y recepción respectivamente, $\lambda$ es la longitud de onda usada y $d$ la distancia entre el transmisor y receptor.

A partir de esta ecuación se puede despejar la distancia teórica entre ambos nodos.

$$d = \sqrt {\frac {G_t \times G_r \times \lambda^2} {4 \times \pi \times P_r} \times P_t}$$

$\lambda$ se puede obtener a partir de la frecuencia ($F$), utilizando la velocidad de la luz ($C$).

$$\lambda = \frac {C} {F}$$

Por otro lado, usando las fórmulas de decibel se puede obtener la potencia ($P_1$) a partir de la intensidad ($L_{dB}$) en relación a una potencia de referencia ($P_0$) que en el caso de las redes inalámbricas es 1mW.

$$P_1 = 10^{\frac {L_{dB}} {10}} \times P_0$$

Finalmente, obtenemos la distancia teórica entre los nodos a partir de la siguiente ecuación.

$$d = \sqrt {\frac {G_t \times G_r \times (\frac {C} {F})^2} {4 \times \pi \times 10^{\frac {L_{dB}} {10}} \times P_0} \times P_t}$$

En donde simplificándola usando los valores antes mencionados se tiene lo siguiente.

$$d = \sqrt {\frac {5 \times 5 \times (\frac {299.792.458} {2.442.000.000})^2} {4 \times \pi \times 10^{\frac {L_{dB}} {10}} \times 0,001} \times P_t}$$

\section{Propuesta}

\subsection{Tecnologías usadas}

Los lenguajes de programación usados serán \emph{C/C++} \cite{language} debido a su sencillez, bajo consumo de memoria, velocidad de ejecución y herramientas disponibles.\\

Se implementará el algoritmo usando el framework \emph{Malva} \cite{malva}, el cual es un fork de \emph{Mallba} \cite{mallba}. Se estudiaron otras alternativas pero finalmente se optó por este debido a recomendaciones hechas en el curso y al hecho de que contábamos con el soporte del foro de discusión del mismo.\\

\subsection{Representación interna}

Cada individuo será una n-úpla de la forma:

$$I = (x, y, d_1, \dots, d_n), x, y \in (-\infty, +\infty), d_i \in [0, 1] \forall i \in (1, \dots, n)$$

En donde \textit{(x, y)} se corresponde con las coordenadas estimadas del individuo en el plano y las constantes $d_i$ representan la atenuación estimada de la señal entre dicho nodo y el router \textit{i}.\\

Por ejemplo, consideremos el siguiente individuo:

$$I = (10, -5, 0.15, 0, 0.1)$$

De acuerdo a la representación descripta, este individuo se estima que está en la posición \textit{(10, -5)} en el plano (con respecto al origen de coordenadas arbitrario usado para inicializar los routers) y que tiene un decaimiento estimado de señal del 15\%, 0\% y 10\% con el router $R_1$, $R_2$ y $R_3$ respectivamente.\\

En otras palabras, si la intensidad de la señal que percibe de, por ejemplo, el router $R_1$ es de 82\%, para el cálculo del fitness se asumirá que la misma debió ser 15\% mayor, por lo cual se usará una intensidad teórica de $82 \times 1.15 = 94\%$.\\

\subsection{Función de fitness}

La función de fitness a minimizar es la suma de errores entre la posición estimada por el algoritmo evolutivo y la distancia teórica obtenida mediante el modelo de propagación \emph{free space propagation model} descripto anteriormente.\\

\begin{dmath*}
$$f(x, y, d_1, ..., d_n) = \sum_{i=1}^{n} |\sqrt{(x-x_o)^2 + (y-y_0)^2} - distance(s_i \times (1 + d_i))|
\end{dmath*}

En donde $s_i$ es la intensidad recibida del router $R_i$.\\

\subsection{Inicialización}

Como se ha dicho antes, se asume que los routers comparten características similares, en particular, que el alcance máximo para todos ellos es de \textit{K} metros.\\

La población será inicializada eligiendo valores aleatorios para \textit{(x, y)} de manera que todos los individuos comiencen  en la intersección de los círculos de centro $(R_{i_x}, R_{i_y})$ y radio \textit{K}, en donde $(R_{i_x}, R_{i_y})$ es la posición del router $R_i$.\\

De esta manera, durante la inicialización de la población, los valores estarán limitados por:

$$x_{max} = rand(\max_{i=1} {R_{i_x} - K}, \min_{i=1} {R_{i_x} + K})$$
$$y_{max} = rand(\max_{i=1} {R_{i_y} - K}, \min_{i=1} {R_{i_y} + K})$$

\subsection{Operadores}

Dado que el espacio de búsqueda es inmenso, resulta esencial mantener la diversidad hasta haber muestreado una amplia área del plano, ya que en principio se puede asumir la existencia de muchos óptimos locales. Dicho esto, los operadores que serán utilizados se describen a continuación.\\

\subsubsection{Selección}

Se usará selección estocástica universal con punteros equiespaciados.\\

\subsubsection{Cruzamiento}

El operador de cruzamiento promediará los valores de ambos padres.

$$x_3 = \frac {x_1 + x_2} {2}, y_3 = \frac {y_1 + y_2} {2}, d_{3_i} = \frac {d_{1_i} + d_{2_i}} {2}$$

\subsubsection{Mutación}

El operador de mutación alterará de acuerdo a un valor aleatorio elegido en un determinado rango.

$$x = x + k_1, y = y + k_2, d_i = d_i + k_{3_i}, k \in (n,m)$$

\subsubsection{Búsqueda Local (sí? no? por qué?)}

\subsection{Mecanismo de reparación}

Cabe señalar que los individuos son válidos desde el momento de su inicialización y los operadores aplicados no los corrompen, por lo que no hace falta un mecanismo de reparación o de descarte de individuos inválidos.

\section{Análisis Experimental}

\subsection{Entorno de prueba}
\subsection{Instancias de prueba}
\subsection{Ajustes de parámetros}
\subsection{Resultados obtenidos}

Para evaluar el algoritmo se harán 30 ejecuciones independientes para el caso de 3, 4, 5, ..., 10 nodos. Adicionalmente se harán 3 ejecuciones en un entorno real, usando 3, 4 y 5 routers.

~\\Para todos los casos se reportará:\\

\begin{itemize}
\item Tiempo de ejecución
\item Generación del mejor individuo
\item Fitness del mejor individuo
\item Valor promedio del fitness
\item Desviación estándar del fitness
\end{itemize}

~\\Para los casos reales se reportará además:\\

\begin{itemize}
\item El error entre la posición del mejor individuo y la posición real que debió encontrar
\item La diferencia entre dicha posición y la posición devuelta por servicios existentes
\end{itemize}

% An example of a floating figure using the graphicx package.
% Note that \label must occur AFTER (or within) \caption.
% For figures, \caption should occur after the \includegraphics.
% Note that IEEEtran v1.7 and later has special internal code that
% is designed to preserve the operation of \label within \caption
% even when the captionsoff option is in effect. However, because
% of issues like this, it may be the safest practice to put all your
% \label just after \caption rather than within \caption{}.
%
% Reminder: the "draftcls" or "draftclsnofoot", not "draft", class
% option should be used if it is desired that the figures are to be
% displayed while in draft mode.
%
%\begin{figure}[!t]
%\centering
%\includegraphics[width=2.5in]{myfigure}
% where an .eps filename suffix will be assumed under latex, 
% and a .pdf suffix will be assumed for pdflatex; or what has been declared
% via \DeclareGraphicsExtensions.
%\caption{Simulation Results.}
%\label{fig_sim}
%\end{figure}

% Note that IEEE typically puts floats only at the top, even when this
% results in a large percentage of a column being occupied by floats.


% An example of a double column floating figure using two subfigures.
% (The subfig.sty package must be loaded for this to work.)
% The subfigure \label commands are set within each subfloat command,
% and the \label for the overall figure must come after \caption.
% \hfil is used as a separator to get equal spacing.
% Watch out that the combined width of all the subfigures on a 
% line do not exceed the text width or a line break will occur.
%
%\begin{figure*}[!t]
%\centering
%\subfloat[Case I]{\includegraphics[width=2.5in]{box}%
%\label{fig_first_case}}
%\hfil
%\subfloat[Case II]{\includegraphics[width=2.5in]{box}%
%\label{fig_second_case}}
%\caption{Simulation results.}
%\label{fig_sim}
%\end{figure*}
%
% Note that often IEEE papers with subfigures do not employ subfigure
% captions (using the optional argument to \subfloat[]), but instead will
% reference/describe all of them (a), (b), etc., within the main caption.


% An example of a floating table. Note that, for IEEE style tables, the 
% \caption command should come BEFORE the table. Table text will default to
% \footnotesize as IEEE normally uses this smaller font for tables.
% The \label must come after \caption as always.
%
%\begin{table}[!t]
%% increase table row spacing, adjust to taste
%\renewcommand{\arraystretch}{1.3}
% if using array.sty, it might be a good idea to tweak the value of
% \extrarowheight as needed to properly center the text within the cells
%\caption{An Example of a Table}
%\label{table_example}
%\centering
%% Some packages, such as MDW tools, offer better commands for making tables
%% than the plain LaTeX2e tabular which is used here.
%\begin{tabular}{|c||c|}
%\hline
%One & Two\\
%\hline
%Three & Four\\
%\hline
%\end{tabular}
%\end{table}

% Note that IEEE does not put floats in the very first column - or typically
% anywhere on the first page for that matter. Also, in-text middle ("here")
% positioning is not used. Most IEEE journals use top floats exclusively.
% Note that, LaTeX2e, unlike IEEE journals, places footnotes above bottom
% floats. This can be corrected via the \fnbelowfloat command of the
% stfloats package.

\section{Conclusión}

The conclusion goes here.


% if have a single appendix:
%\appendix[Proof of the Zonklar Equations]
% or
%\appendix  % for no appendix heading
% do not use \section anymore after \appendix, only \section*
% is possibly needed

% use appendices with more than one appendix
% then use \section to start each appendix
% you must declare a \section before using any
% \subsection or using \label (\appendices by itself
% starts a section numbered zero.)
%


\appendices
\section{Proof of the First Zonklar Equation}
Appendix one text goes here.

% Can use something like this to put references on a page
% by themselves when using endfloat and the captionsoff option.
\ifCLASSOPTIONcaptionsoff
  \newpage
\fi



% trigger a \newpage just before the given reference
% number - used to balance the columns on the last page
% adjust value as needed - may need to be readjusted if
% the document is modified later
%\IEEEtriggeratref{8}
% The "triggered" command can be changed if desired:
%\IEEEtriggercmd{\enlargethispage{-5in}}

% references section

% can use a bibliography generated by BibTeX as a .bbl file
% BibTeX documentation can be easily obtained at:
% http://www.ctan.org/tex-archive/biblio/bibtex/contrib/doc/
% The IEEEtran BibTeX style support page is at:
% http://www.michaelshell.org/tex/ieeetran/bibtex/
%\bibliographystyle{IEEEtran}
% argument is your BibTeX string definitions and bibliography database(s)
%\bibliography{IEEEabrv,../bib/paper}
%
% <OR> manually copy in the resultant .bbl file
% set second argument of \begin to the number of references
% (used to reserve space for the reference number labels box)

\bibliographystyle{unsrt}
\bibliography{refs}

% You can push biographies down or up by placing
% a \vfill before or after them. The appropriate
% use of \vfill depends on what kind of text is
% on the last page and whether or not the columns
% are being equalized.

%\vfill

% Can be used to pull up biographies so that the bottom of the last one
% is flush with the other column.
%\enlargethispage{-5in}



% that's all folks
\end{document}